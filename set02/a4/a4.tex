\documentclass[11pt,a4paper]{scrarticle}
\usepackage[utf8]{inputenc}
\usepackage{cmap}
\usepackage[T2A]{fontenc}
\usepackage[russian]{babel}
\usepackage{amsmath,amssymb,amsthm,mathtools}
\usepackage{array}

\usepackage{indentfirst}
\usepackage{xcolor,graphicx, tikz, wrapfig}
\usepackage{longtable}

\usepackage{minted}
\usemintedstyle{vs}

\usepackage[left=2cm,right=2cm,top=2cm,bottom=2cm,bindingoffset=0cm]{geometry}

\usepackage[unicode]{hyperref}
\definecolor{linkcolor}{HTML}{0000E6}
\definecolor{urlcolor}{HTML}{0000E6}
\definecolor{citecolor}{HTML}{0000E6}
\hypersetup{pdfpagemode=None,linktoc=page,citecolor=citecolor,linkcolor=linkcolor,urlcolor=urlcolor,colorlinks=true}

\theoremstyle{definition}
\newtheorem{subtask}{Пункт}

\DeclareMathOperator*{\argmax}{arg\,max}
\DeclareMathOperator*{\argmin}{arg\,min}
\newcommand{\floor}[1]{\left\lfloor #1 \right\rfloor}
\newcommand{\ceil}[1]{\left\lceil #1 \right\rceil}


\setlength{\parindent}{1cm}

\author{Клычков Максим Дмитриевич}

\begin{document}

\centerline{\textbf{\huge Алгоритмы и структуры данных-1}}
\centerline{\textbf{SET 2. Задача A4.}}
\begin{flushright}
    \emph{Осень 2024. Клычков М. Д.}
\end{flushright}

\begin{subtask}
    Представим описание алгоритма в виде программы на \texttt{C++}:

    \begin{figure}[h]
        \centering
        \begin{minted}[linenos]{cpp}
long long CINV(std::vector<long long> &arr) {
    // Base case
    if (arr.size() == 1) {
        return 0;
    }

    // 1. Divide
    size_t mid = arr.size() / 2;
    std::vector<long long> left(arr.begin(), arr.begin() + mid);
    std::vector<long long> right(arr.begin() + mid, arr.end());

    // 2. Conquer
    auto count_left = CINV(left);
    auto count_right = CINV(right);
    
    // 3. Combine
    long long count_merge{};
    for (auto iter = arr.begin(), iter_left = left.begin(), iter_right = right.begin();
            iter != arr.end(); ++iter) {
        if (iter_left != left.end() &&
              (iter_right == right.end() || *iter_left <= *iter_right)) {
            *iter = *iter_left;
            ++iter_left;
        } else {
            *iter = *iter_right;
            count_merge += std::distance(iter_left, left.end());
            ++iter_right;                
        }
    }

    return count_left + count_right + count_merge;
}
        \end{minted}
        \caption{\label{fig:CINV} \texttt{CINV} algorithm}
    \end{figure}

    Неформально опишу, что происходит в этом алгоритме. Для того чтобы посчитать количество инверсий, здесь применяется принцип \emph{разделяй и властвуй}, а именно «модифицированный» алгоритм \texttt{MergeSort}. Помимо сортировки на каждом этапе \texttt{COMBINE} мы подсчитываем количество инверсий, а ответом будет являться их сумма.

    Как же вычисляется это количество? Представим, что при объединении двух упорядоченных подмассивов в один (\emph{merge}) текущий элемент из правого подмассива меньше, чем текущий элемент левого подмассива. Тогда оставшиеся (еще не обработанные) и само текущее числа левого подмассива образуют инверсию с рассматриваемым правым. Найденное число инверсий прибавляется к общему (строка $26$ кода).

    При таком подходе сохранятся все инверсии, которые образовывались элементами из разных подмассивов, поэтому искомое число инверсий будет получено.

    Опишем суть шагов \texttt{DIVIDE}, \texttt{CONQUER} и \texttt{COMBINE}, характерных для \texttt{DaC} алгоритмов.

    \begin{enumerate}
        \item \texttt{DIVIDE}: Разделяем массив на два подмассива (если число элементов нечетное, количество элементов в правом подмассиве больше на $1$).
        \item \texttt{CONQUER}: Рекурсивно применяем алгоритм для полученных на этапе \texttt{DIVIDE} подмассивах. Сохраняем результат -- количество инверсий в левой и правой половинах. Базовый случай (\emph{остановка рекурсии}) происходит, когда длина подмассива становится равной единице, в таком случае количество инверсий равно $0$.
        \item \texttt{COMBINE}: Считаем количество инверсий между двумя упорядоченными подмассивами и собираем из них упорядоченный исходный массив. Если текущий элемент левого подмассива меньше, то он копируется в исходный массив на нужную позицию, но если элемент из правого подмассива больше, то помимо добавления в начальный массив, происходит увеличение счетчика инверсий. Как именно это происходит и почему это верно описано выше.
    \end{enumerate}

    Выразим рекуррентное соотношение, описывающее функцию временной сложности $T(n)$. В рекурсивной ветке вычислений происходит два вызова подзадача вдвое меньшего размера, а в нерекурсивной -- \emph{merge} двух подмассивов, производящийся проходом цикла до $n$. Тогда можем записать (опускаем округление вниз):
    \begin{equation}
        \label{eqn:rec}
        T(n) = 2T\left(\frac{n}{2}\right) + O(n)
    \end{equation}

    Воспользуемся мастер-теоремой: $a = 2, b = 2, k = 1, f(n) = 1, \log_b a = 1$. Тогда нам подходит случай $k = log_b a$, следовательно получим:
    \begin{equation}
        \label{eqn:bigoh}
        T(n) = O(n \cdot \log_2 n)
    \end{equation}

    Стоит отметить, что разработанный алгоритм изменяет исходный массив, поэтому если есть необходимость сохранить начальный порядок, асимптотика не изменится, так как \linebreak $T(n) = O(n \cdot \log_2 n) + O(n) = O(n \cdot \log_2 n)$

    Заметим, что полученное общее число инверсий не всегда равно минимальному числу, достаточному для получения отсортированного массива. Например, в массиве $A = [5, 3, 2, 4, 1]$ всего $8$ инверсий, однако, чтобы получить отсортированный массив достаточно сделать только две: $5 \leftrightarrow 1$ и $3 \leftrightarrow 2$.
\end{subtask}

\begin{subtask}
    Сохраним общую схему решения, основанную на модификации \texttt{MergeSort}. Хотим, чтобы шаги \texttt{DIVIDE} и \texttt{CONQUER} должны остаться неизменными, попытаемся изменить только \texttt{COMBINE}.

    Чтобы суть алгоритма (\emph{поиск числа инверсий между двумя половинами}) осталась неизменной, попробуем сохранить встроенную в этот алгоритм сортировку. Но уже не получится так легко подсчитывать инверсии в одно и то же время, что и \emph{merge} двух половин. Тогда добавим дополнительный проход, который будет считать только инверсии, а за ним будет уже соединение двух упорядоченных подмассивов.

    Таким образом в нерекурсивную ветку вычислений рекурренты добавится терм $n$, но он никак не повлияет на асимптотику:
    \begin{equation}
        T'(n) = T(n) + n = 2T\left(\frac{n}{2}\right) + O(n) + n = 2T\left(\frac{n}{2}\right) + O(n) = O(n \cdot \log_2 n)
    \end{equation}

    Также опишем этот алгоритм кодом \ref{fig:CSINV}

    \begin{figure}[h]
        \centering
        \begin{minted}[linenos]{cpp}
long long CSINV(std::vector<long long> &arr) {
    // Base case
    if (arr.size() == 1) {
        return 0;
    }

    // 1. Divide
    size_t mid = arr.size() / 2;
    std::vector<long long> left(arr.begin(), arr.begin() + mid);
    std::vector<long long> right(arr.begin() + mid, arr.end());

    // 2. Conquer
    auto count_left = CSINV(left);
    auto count_right = CSINV(right);
    
    // 3. Combine

    // 3.1. Combine - Count inversions which starts at left half and ends at right part 
    long long count_merge{};
    for (auto iter_left = left.begin(), iter_right = right.begin();
            iter_left != left.end() && iter_right != right.end();) {
        if (iter_left != left.end() &&
              (iter_right == right.end() || *iter_left <= 2 * *iter_right)) {
            ++iter_left;
        } else {
            count_merge += std::distance(iter_left, left.end());
            ++iter_right;                
        }
    }

    // 3.2. Combine - Merge two halfs to make array sorted
    for (auto iter = arr.begin(), iter_left = left.begin(), iter_right = right.begin();
            iter != arr.end(); ++iter) {
        if (iter_left != left.end() &&
              (iter_right == right.end() || *iter_left <= *iter_right)) {
            *iter = *iter_left;
            ++iter_left;
        } else {
            *iter = *iter_right;
            ++iter_right;
        }
    }
    
    return count_left + count_right + count_merge;
}
        \end{minted}
        \caption{\label{fig:CSINV} \texttt{CSINV} algorithm}
    \end{figure}

\end{subtask}

\end{document}