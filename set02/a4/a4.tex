\documentclass[11pt,a4paper]{scrarticle}
\usepackage[utf8]{inputenc}
\usepackage{cmap}
\usepackage[T2A]{fontenc}
\usepackage[russian]{babel}
\usepackage{amsmath,amssymb,amsthm,mathtools}
\usepackage{array}

\usepackage{indentfirst}
\usepackage{xcolor,graphicx, tikz, wrapfig}
\usepackage{longtable}

% \usepackage{minted}
% \usemintedstyle{vs}

\usepackage[left=2cm,right=2cm,top=2cm,bottom=2cm,bindingoffset=0cm]{geometry}

\usepackage[unicode]{hyperref}
\definecolor{linkcolor}{HTML}{0000E6}
\definecolor{urlcolor}{HTML}{0000E6}
\definecolor{citecolor}{HTML}{0000E6}
\hypersetup{pdfpagemode=None,linktoc=page,citecolor=citecolor,linkcolor=linkcolor,urlcolor=urlcolor,colorlinks=true}

\theoremstyle{definition}
\newtheorem{subtask}{Пункт}

\DeclareMathOperator*{\argmax}{arg\,max}
\DeclareMathOperator*{\argmin}{arg\,min}
\newcommand{\floor}[1]{\left\lfloor #1 \right\rfloor}
\newcommand{\ceil}[1]{\left\lceil #1 \right\rceil}


\setlength{\parindent}{1cm}

\author{Клычков Максим Дмитриевич}

\begin{document}

\centerline{\textbf{\huge Алгоритмы и структуры данных-1}}
\centerline{\textbf{SET 2. Задача A4.}}
\begin{flushright}
  \emph{Осень 2024. Клычков М. Д.}
\end{flushright}

\begin{subtask}
  Для начала стоит вспомнить, что нам известно об асимптотике алгоритма Штрассена. На лекции было доказано, что $T_\text{Ш} = O(n^{\log_2 7})$. От нас требуется найти алгоритм, выражающийся рекуррентным соотношением
  \begin{equation}
    \label{eqn:t}
    T(n) = a \cdot T\left(\frac{n}{4}\right) + \Theta(n^2)
  \end{equation}

  Заметим, что для данного рекуррентного соотношения возможно применение мастер-теоремы.

  \begin{align}
    \label{eqn:case1} \text{при } \log_4 a = 2 \Leftrightarrow a = 16 & \colon T(n) = O(n^2 \log_4 n) \\
    \label{eqn:case2} \text{при } \log_4 a > 2 \Leftrightarrow a > 16 & \colon T(n) = O(n^{\log_4 a}) \\
    \label{eqn:case3} \text{при } \log_4 a < 2 \Leftrightarrow a < 16 & \colon T(n) = O(n^2)
  \end{align}


  На лекции отмечалось, что умножение квадратных матриц может быть реализовано только за $O(n^{2+\varepsilon}), \, \varepsilon > 0$. Воспользуемся этой информацией и уберем из рассмотрения \eqref{eqn:case3}, так как в этом случае $T(n) = O(n^2)$.

  Покажем, что нам подходит случай \eqref{eqn:case1}. Достаточно, чтобы $T(n) = O(T_\text{Ш}(n))$ или, подставив конкретную $T(n)$, $O(n^2 \log_4 n) = O(n^{\log_2 7})$

  \begin{gather*}
    n^2 \log_4 n \leq c n^{\log_2 7} \\
    \log_4 n \leq c n^{\log_2 7 - 2} \\
    \log_4 n \leq c n^{\log_2 1.75} \Leftrightarrow \log n = O(n^{\log_2 1.75})
  \end{gather*}

  А последнее мы принимаем за истину (знания о порядке роста функций).

  Рассмотрим оставшийся случай \eqref{eqn:case2}. Тут все действия будут схожи с предыдущим пунктом, только остается решить уравнение с параметром.
  Нам нужно найти такие $a$, что $T(n) = O(T_\text{Ш}(n))$ или, подставив конкретную $T(n)$, $O(n^{\log_4 a}) = O(n^{\log_2 7}) \Leftrightarrow \log_4 a \leq \log_2 7$

  \begin{gather*}
    \frac{1}{2} \log_2 a \leq \log_2 7 \\
    \log_2 a \leq \log_2 49 \\
    a \leq 49
  \end{gather*}

  % дописать ответ !!!!!!!!!!!!!!!!!!!!!!!!!!!!!!!!!!!!!!!!!!!!!!!!!!!!!!!!!!!!!!!
  % включать или нет 49 !!!!!!!!!!!!!!!!!!!!!!!!!!!!!!!!!!!!!!!!!!!!!!!!
  % пользоваться ли 2+eps ??????????????????????????????????????????????????????
\end{subtask}

\end{document}