\documentclass[11pt,a4paper]{scrarticle}
\usepackage[utf8]{inputenc}
\usepackage{cmap}
\usepackage[T2A]{fontenc}
\usepackage[russian]{babel}
\usepackage{amsmath,amssymb,amsthm,mathtools}
\usepackage{array}

\usepackage{indentfirst}
\usepackage{xcolor,graphicx, tikz, wrapfig}
\usepackage{longtable}
\usepackage{placeins}

\usepackage{minted}
\usemintedstyle{vs}

\usepackage[left=2cm,right=2cm,top=2cm,bottom=2cm,bindingoffset=0cm]{geometry}

\usepackage[unicode]{hyperref}
\definecolor{linkcolor}{HTML}{0000E6}
\definecolor{urlcolor}{HTML}{0000E6}
\definecolor{citecolor}{HTML}{0000E6}
\hypersetup{pdfpagemode=None,linktoc=page,citecolor=citecolor,linkcolor=linkcolor,urlcolor=urlcolor,colorlinks=true}

\theoremstyle{definition}
\newtheorem{subtask}{Пункт}

\DeclareMathOperator*{\argmax}{arg\,max}
\DeclareMathOperator*{\argmin}{arg\,min}
\newcommand{\floor}[1]{\left\lfloor #1 \right\rfloor}
\newcommand{\ceil}[1]{\left\lceil #1 \right\rceil}


\setlength{\parindent}{1cm}

\author{Клычков Максим Дмитриевич}

\begin{document}

\centerline{\textbf{\huge Алгоритмы и структуры данных-1}}
\centerline{\textbf{SET 3. Задача A3.}}
\begin{flushright}
	\emph{Осень 2024. Клычков М. Д.}
\end{flushright}

\begin{itemize}
	\item ID посылки по задаче \texttt{A1i}: \href{https://dsahse.contest.codeforces.com/group/NOflOR1Qt0/contest/565612/submission/292666711}{292666711}
	\item Ссылка на репозиторий:
\end{itemize}

\subsection*{Внутренняя инфраструктура для экспериментального анализа}

Внутренняя инфраструктура для экспериментального анализа практически полностью была скопирована из предыдущей задачи. Из изменений добавился только общий генератор случайных чисел, распространяющийся на весь проект (это позволительно, так как мы запускаем программу в одном потоке).

Были реализованы классы \texttt{ArrayGenerator} и \texttt{SortTester} для организации удобного замера работы алгоритмов.

\begin{figure}[htp]
	\centering
	\inputminted[linenos,fontsize=\small]{cpp}{../analyze/generator.h}
	\caption{\texttt{ArrayGenerator Header File}}
	\label{code:generator-h}
\end{figure}
\FloatBarrier

\begin{figure}[htp]
	\centering
	\inputminted[linenos,fontsize=\small]{cpp}{../analyze/generator.cpp}
	\caption{\texttt{ArrayGenerator Cpp File}}
	\label{code:generator-cpp}
\end{figure}
\FloatBarrier

\begin{figure}[htp]
	\centering
	\inputminted[linenos,fontsize=\small]{cpp}{../analyze/tester.h}
	\caption{\texttt{SortTester Header File}}
	\label{code:tester-h}
\end{figure}
\FloatBarrier

\begin{figure}[htp]
	\centering
	\inputminted[linenos,fontsize=\small]{cpp}{../analyze/tester.cpp}
	\caption{\texttt{SortTester Cpp File}}
	\label{code:tester-cpp}
\end{figure}
\FloatBarrier

\begin{figure}[htp]
	\centering
	\inputminted[linenos,fontsize=\small]{cpp}{../analyze/random_utils.h}
	\caption{\texttt{Random Utils Header File}}
	\label{code:random-h}
\end{figure}
\FloatBarrier

В функции \texttt{main()} происходит создание сразу трех объектов \texttt{SortTester} с разными \texttt{seed}. Каждый запуск происходит по три раза на каждом из тестеров, затем результаты усредняются.

\emph{Некоторые замечания}: эта реализация не является оптимальной по крайней мере потому, что один и тот же код вызова сортировок повторятся для каждого алгоритма, однако было принято решение оставить такую реализацию, так как она уж точно не содержит никаких подводных камней, которые могут повлиять на временную оценку.

\subsection*{Анализ}

Для построения графиков использовался язык \texttt{Python}.

Сначала проанализируем каждый алгоритм сортировки по отдельности. Начнем с \texttt{QuickSort}.

\begin{figure}[htp]
	\centering
	\includegraphics[width=\textwidth]{../static/quick.png}
	\caption{\texttt{QuickSort}}
	\label{fig:quick}
\end{figure}
\FloatBarrier

Можем заметить, что сортировка случайно заполненного неупорядоченного массива $\approx 1.33$~раза больше, чем сортировка почти упорядоченного по неубыванию и упорядоченного по невозрастанию массивов, время сортировки которых при равном $n$ приблизительно равны.

Казалось бы, сортировка почти упорядоченного массива должна лидировать по времени исполнения, однако при случайно выбранном элементе \texttt{pivot} это не так.

Рассмотрим \texttt{IntroSort}.

\begin{figure}[htp]
	\centering
	\includegraphics[width=\textwidth]{../static/intro.png}
	\caption{\texttt{IntroSort}}
	\label{fig:quick}
\end{figure}
\FloatBarrier

На этом графике наблюдается та же ситуация --- случайно заполненный массив «проигрывает» по времени двум другим видам массивов. Также здесь наблюдается более сильное расхождение по времени между обратно отсортированными и почти отсортированными массивами. Можно выдвинуть предположение, что такое влияние оказывает встроенный в \texttt{IntroSort} \texttt{InsertionSort}.

Теперь же перейдем к сравнению двух алгоритмов: \texttt{IntroSort} и \texttt{QuickSort}.

\begin{figure}[htp]
	\centering
	\includegraphics[width=0.85\textwidth]{../static/array_types.png}
	\caption{\texttt{IntroSort} \emph{vs.} \texttt{QuickSort}}
	\label{fig:types}
\end{figure}
\FloatBarrier

Результаты \emph{поражающие}! В каждом из случаев \texttt{IntroSort} показывает результат лучше другого алгоритма.

\section*{Выводы}

Основные выводы по графикам были сделаны выше, тут же только отметим, что  при выборе между этими двумя сортировками стоит использовать вариант \texttt{IntroSort}.

\end{document}

