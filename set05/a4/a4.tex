\documentclass[11pt,a4paper]{scrarticle}
\usepackage[utf8]{inputenc}
\usepackage{cmap}
\usepackage[T2A]{fontenc}
\usepackage[russian]{babel}
\usepackage{amsmath,amssymb,amsthm,mathtools}
\usepackage{array}

\usepackage{indentfirst}
\usepackage{xcolor,graphicx, tikz, wrapfig}
\usepackage{longtable}
\usepackage{placeins}

\usepackage{minted}
\usemintedstyle{vs}

\usepackage[left=2cm,right=2cm,top=2cm,bottom=2cm,bindingoffset=0cm]{geometry}

\usepackage[unicode]{hyperref}
\definecolor{linkcolor}{HTML}{0000E6}
\definecolor{urlcolor}{HTML}{0000E6}
\definecolor{citecolor}{HTML}{0000E6}
% \hypersetup{pdfpagemode=None,linktoc=page,citecolor=citecolor,linkcolor=linkcolor,urlcolor=urlcolor,colorlinks=true}

\theoremstyle{definition}
\newtheorem{subtask}{Пункт}

\DeclareMathOperator*{\argmax}{arg\,max}
\DeclareMathOperator*{\argmin}{arg\,min}
\newcommand{\floor}[1]{\left\lfloor #1 \right\rfloor}
\newcommand{\ceil}[1]{\left\lceil #1 \right\rceil}


\setlength{\parindent}{1cm}

\author{Клычков Максим Дмитриевич}

\begin{document}

\centerline{\textbf{\huge Алгоритмы и структуры данных-2}}
\centerline{\textbf{SET 5. Задача A4.}}
\begin{flushright}
    \emph{Весна 2024. Клычков М. Д.}
\end{flushright}


\begin{subtask}
    Пусть $n$ — количество хеш-функций обоих фильтров Блума. Пусть хеш-функции $H = h_1, h_2, \dots, h_n \in \mathcal{H}$ для фильтров Блума $F(A)$ и $F(B)$ одинаковы (при различных функциях ответ на поставленный вопрос очевиден). Пусть $A, B$ — битовые массивы, стоящие за фильтрами $F(A), F(B)$ соответственно, $A_j, B_j, j = \overline{1, m}$ — биты, где $m$ — количество битов в фильтрах.

    Докажем, что $F(AB), (AB)_j = A_j \& B_j$ будет выдавать положительные ответы о принадлежности объектов из множества $A \cap B$.

    \begin{proof}
        Предположим обратное: $\exists x \in A \cap B \colon F(AB) = 0 \Leftrightarrow \exists j \in [1, m] \colon (AB)_j = 0 \Leftrightarrow  A_j \& B_j = 0$. Тогда $A_j = 0 \lor B_j = 0 \Leftrightarrow x \notin A \lor x \notin B \Leftrightarrow x \notin A \cap B$. Получили противоречие.
    \end{proof}
\end{subtask}

\begin{subtask}
    Покажем, что $F(AB)$ не обязательно в точности соответствует другому фильтру, который будет получен в результате последовательной вставки объектов из множества $A \cap B$ (обозначим такой фильтр за $F'(AB)$, а его битовый массив за $(AB)'$).

    \begin{proof}
        Зафиксируем два множества $A = \{e_A, e_1, e_2, e_3, \dots\}$ и $B = \{e_B, e_1, e_2, e_3, \dots\}$, здесь \linebreak $\{e_1, e_2, e_3, \dots\} \in A \cap B$, причем элементы подберем таким образом, чтобы выполнялись дополнительные условия (оставим вне рассмотрения причину, почему такие элементы вообще существуют, очевидно):
        \begin{align*}
            \forall h \in H \forall e_1, e_2, e_3, \dots \colon (h(e_i))_{j_0} = 0 \\
            \exists h \in H \colon (h(e_A))_{j_0} = 1                              \\
            \exists h \in H \colon (h(e_B))_{j_0} = 1                              \\
        \end{align*}

        Другими словами, элементы, принадлежащие пересечению множеств $A \cap B$ не «вносят вклад» в бит $j_0$, но находятся элементы $e_A, e_B \notin A \cap B$ такие, что «включают» бит $j_0$.

        Из фильтров Блума $F(A)$ и $F(B)$, получим фильтр $F(AB)$, в нем бит $(AB)_{j_0} = 1$.

        Построим новый фильтр Блума $F'(AB)$, последовательно добавляя элементы $e_1, e_2, e_3, \dots \in A \cap B$. Но тогда в таком фильтре Блума $(AB)'_{j_0} = 0$ согласно условию выше, поэтому получаем, что $(AB) \neq (AB)'$ как массивы одинакового размера, то есть $F(AB) \neq F'(AB)$
    \end{proof}
\end{subtask}

\end{document}
