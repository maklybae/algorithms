\documentclass[11pt,a4paper]{scrarticle}
\usepackage[utf8]{inputenc}
\usepackage{cmap}
\usepackage[T2A]{fontenc}
\usepackage[russian]{babel}
\usepackage{amsmath,amssymb,amsthm,mathtools}
\usepackage{array}

\usepackage{indentfirst}
\usepackage{xcolor,graphicx, tikz, wrapfig}
\usepackage{longtable}
\usepackage{multicol}
\usepackage{placeins}

\usepackage{minted}
\usemintedstyle{vs}

\usepackage[left=2cm,right=2cm,top=2cm,bottom=2cm,bindingoffset=0cm]{geometry}

\usepackage[unicode]{hyperref}
\definecolor{linkcolor}{HTML}{0000E6}
\definecolor{urlcolor}{HTML}{0000E6}
\definecolor{citecolor}{HTML}{0000E6}
% \hypersetup{pdfpagemode=None,linktoc=page,citecolor=citecolor,linkcolor=linkcolor,urlcolor=urlcolor,colorlinks=true}

\theoremstyle{definition}
\newtheorem{subtask}{Пункт}

\DeclareMathOperator*{\argmax}{arg\,max}
\DeclareMathOperator*{\argmin}{arg\,min}
\newcommand{\floor}[1]{\left\lfloor #1 \right\rfloor}
\newcommand{\ceil}[1]{\left\lceil #1 \right\rceil}


\setlength{\parindent}{1cm}

\author{Клычков Максим Дмитриевич}

\begin{document}

\centerline{\textbf{\huge Алгоритмы и структуры данных-2}}
\centerline{\textbf{SET 5. Задача A3b.}}
\begin{flushright}
    \emph{Весна 2024. Клычков М. Д.}
\end{flushright}

\setlength{\tabcolsep}{4pt}
\renewcommand{\arraystretch}{1.2}

\begin{subtask}
    Будем работать только с двухсимвольными строками. Пусть $s_1, s_2$ — первый и второй символы строки соответственно, $f(s_1), f(s_2)$ — \texttt{ASCII} коды первого и второго символа соответственно, $g(s_i) = f(s_i) - f(\text{'a'}) + 1$ — смещенный код $i$-го символа строки.

    Тогда хеш-функцию можно записать как $h(s) = g(s_1) + g(s_2) p$. Требуется найти такие строки (формально вектор из двух символов), что $h(s) = 0$, то есть
    $$g(s_1) + g(s_2) p = 0 \Longleftrightarrow -g(s_1) = g(s_2) p.$$

    Алгоритм поиска описанных строк при заданном параметре $p$ будет заключаться в переборе всех возможных пар символов и проверке условия $-g(s_1) = g(s_2) p$. Заметим, что мы за при просмотре строки-пары $s = \overline{s_1 s_2}$ мы также можем рассмотреть строку $s' = \overline{s_2 s_1}$. Для понимания, как будут выглядеть значения $g(s_i)$ выпишем некоторые из них:

    \begin{table}[h!]
        \centering
        \begin{tabular}{|c|c|c|}
            \hline
            $s_i$  & $f(s_i)$ & $g(s_i)$ \\
            \hline
            0      & 48       & -48      \\
            1      & 49       & -47      \\
            2      & 50       & -46      \\
            \vdots & \vdots   & \vdots   \\
            9      & 57       & -39      \\
            A      & 65       & -31      \\
            B      & 66       & -30      \\
            C      & 67       & -29      \\
            \vdots & \vdots   & \vdots   \\
            Z      & 90       & -6       \\
            a      & 97       & 1        \\
            b      & 98       & 2        \\
            c      & 99       & 3        \\
            \vdots & \vdots   & \vdots   \\
            z      & 122      & 26       \\
            \hline
        \end{tabular}
    \end{table}

    Теперь все готово к написанию алгоритма, воспользуемся языком программирования \texttt{Python}:
    \begin{minted}
        [
        frame=lines,
        framesep=2mm,
        baselinestretch=1.2,
        fontsize=\footnotesize,
        linenos
        ]
        {python}
from string import ascii_lowercase, ascii_uppercase, digits

def find_kernel(p: int) -> list[str]:
    symbols = list(digits) + list(ascii_uppercase) + list(ascii_lowercase) # list of symbols
    g = lambda si: ord(si) - ord('a') + 1
    kernel = []
    for i in range(len(symbols) - 1):
        for j in range(i + 1, len(symbols)):
            if -g(symbols[i]) == g(symbols[j]) * p:
                kernel.append(symbols[i] + symbols[j])
            if -g(symbols[j]) == g(symbols[i]) * p:
                kernel.append(symbols[j] + symbols[i])

    return kernel
    \end{minted}

    Проверим работу алгоритма:
    \begin{minted}
        [
        frame=lines,
        framesep=2mm,
        baselinestretch=1.2,
        fontsize=\footnotesize,
        linenos
        ]
        {python}
find_kernel(13) # returns ['9c', 'Fb', 'Sa']
find_kernel(17) # returns ['Oa']
find_kernel(23) # returns ['2b', 'Ia']
    \end{minted}
\end{subtask}

\begin{subtask}
    Напишем код, который, пользуясь написанной функцией, выведет требуемые строки для всех значений $p \le 31$

    \begin{minted}
        [
        frame=lines,
        framesep=2mm,
        baselinestretch=1.2,
        fontsize=\footnotesize,
        linenos
        ]
        {python}
for p in range(-10000, 32):
    ker = find_kernel(p)
    if ker:
        print(p, ker)
    \end{minted}

    Утверждается, что порогового значения $-10000$ для параметра $p$ будет достаточно, так как значения $g(s_i) \in [-48, 26]$
\end{subtask}

Вывод:
\begin{longtable}{|c|c|}
    \hline
    \textbf{p} & \textbf{Values} \\
    \hline
    \endfirsthead
    \hline
    \textbf{p} & \textbf{Values} \\
    \hline
    \endhead
    \hline
    \textbf{p} & \textbf{Values} \\
    \hline
    \endfirsthead
    \hline
    \textbf{p} & \textbf{Values} \\
    \hline
    \endhead
    -26        & za              \\
    \hline
    -25        & ya              \\
    \hline
    -24        & xa              \\
    \hline
    -23        & wa              \\
    \hline
    -22        & va              \\
    \hline
    -21        & ua              \\
    \hline
    -20        & ta              \\
    \hline
    -19        & sa              \\
    \hline
    -18        & ra              \\
    \hline
    -17        & qa              \\
    \hline
    -16        & pa              \\
    \hline
    -15        & oa              \\
    \hline
    -14        & na              \\
    \hline
    -13        & ma              \\ & zb \\
    \hline
    -12        & la              \\ & xb \\
    \hline
    -11        & ka              \\ & vb \\
    \hline
    -10        & ja              \\ & tb \\
    \hline
    -9         & ia              \\ & rb \\
    \hline
    -8         & 0Z              \\ & ha \\ & pb \\ & xc \\
    \hline
    -7         & 6Z              \\ & ga \\ & nb \\ & uc \\
    \hline
    -6         & 0X              \\ & 6Y \\ & fa \\ & lb \\ & rc \\ & xd \\
    \hline
    -5         & 3W              \\ & 8X \\ & BZ \\ & ea \\ & jb \\ & oc \\ & td \\ & ye \\
    \hline
    -4         & 0T              \\ & 4U \\ & 8V \\ & DY \\ & HZ \\ & da \\ & hb \\ & lc \\ & pd \\ & te \\ & xf \\
    \hline
    -3         & 0P              \\ & 3Q \\ & 6R \\ & 9S \\ & BV \\ & EW \\ & HX \\ & KY \\ & NZ \\ & ca \\ & fb \\ & ic \\ & ld \\ & oe \\ & rf \\ & ug \\ & xh \\
    \hline
    -2         & 0H              \\ & 2I \\ & 4J \\ & 6K \\ & 8L \\ & BQ \\ & DR \\ & FS \\ & HT \\ & JU \\ & LV \\ & NW \\ & PX \\ & RY \\ & TZ \\ & ba \\ & db \\ & fc \\ & hd \\ & je \\ & lf \\ & ng \\ & ph \\ & ri \\ & tj \\ & vk \\ & xl \\ & zm \\
    \hline
    1          & Fz              \\ & zF \\ & Gy \\ & yG \\ & Hx \\ & xH \\ & Iw \\ & wI \\ & Jv \\ & vJ \\ & Ku \\ & uK \\ & Lt \\ & tL \\ & Ms \\ & sM \\ & Nr \\ & rN \\ & Oq \\ & qO \\ & Pp \\ & pP \\ & Qo \\ & oQ \\ & Rn \\ & nR \\ & Sm \\ & mS \\ & Tl \\ & lT \\ & Uk \\ & kU \\ & Vj \\ & jV \\ & Wi \\ & iW \\ & Xh \\ & hX \\ & Yg \\ & gY \\ & Zf \\ & fZ \\
    \hline
    2          & 0x              \\ & 2w \\ & 4v \\ & 6u \\ & 8t \\ & Bo \\ & Dn \\ & Fm \\ & Hl \\ & Jk \\ & Lj \\ & Ni \\ & Ph \\ & Rg \\ & zS \\ & Tf \\ & xT \\ & vU \\ & Ve \\ & tV \\ & rW \\ & Xd \\ & pX \\ & nY \\ & Zc \\ & lZ \\
    \hline
    3          & 0p              \\ &3o\\ &6n\\ &9m\\ &Bj\\ &Ei\\ &Hh\\ &Kg\\ &Nf\\ &Qe\\ &Td\\ &Wc\\ &xX\\ &uY\\ &Zb\\ &rZ \\
    \hline
    4          & 0l              \\ &4k\\ &8j\\ &Dg\\ &Hf\\ &Le\\ &Pd\\ &Tc\\ &Xb\\ &xZ\\
    \hline
    5          & 3i              \\ &8h\\ &Bf\\ &Ge\\ &Ld\\ &Qc\\ &Vb\\
    \hline
    6          & 0h              \\ &6g\\ &Be\\ &Hd\\ &Nc\\ &Tb\\ &Za\\
    \hline
    7          & 6f              \\ &Dd\\ &Kc\\ &Rb\\ &Ya\\
    \hline
    8          & 0f              \\ &8e\\ &Hc\\ &Pb\\ &Xa\\
    \hline
    9          & 3e              \\ &Ec\\ &Nb\\ &Wa\\
    \hline
    10         & 8d              \\ &Bc\\ &Lb\\ &Va\\
    \hline
    11         & 4d              \\ &Jb\\ &Ua\\
    \hline
    12         & 0d              \\ &Hb\\ &Ta\\
    \hline
    13         & 9c              \\ &Fb\\ &Sa\\
    \hline
    14         & 6c              \\ &Db\\ &Ra\\
    \hline
    15         & 3c              \\ &Bb\\ &Qa\\
    \hline
    16         & 0c              \\ &Pa\\
    \hline
    17         & Oa              \\
    \hline
    18         & Na              \\
    \hline
    19         & Ma              \\
    \hline
    20         & 8b              \\ &La\\
    \hline
    21         & 6b              \\ &Ka\\
    \hline
    22         & 4b              \\ &Ja\\
    \hline
    23         & 2b              \\ &Ia\\
    \hline
    24         & 0b              \\ &Ha\\
    \hline
    25         & Ga              \\
    \hline
    26         & Fa              \\
    \hline
    27         & Ea              \\
    \hline
    28         & Da              \\
    \hline
    29         & Ca              \\
    \hline
    30         & Ba              \\
    \hline
    31         & Aa              \\
    \hline
\end{longtable}
\end{document}

