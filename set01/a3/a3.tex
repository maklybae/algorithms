\documentclass[11pt,a4paper]{scrarticle}
\usepackage[utf8]{inputenc}
\usepackage{cmap}
\usepackage[T2A]{fontenc}
\usepackage[russian]{babel}
\usepackage{amsmath,amssymb,amsthm,mathtools}
\usepackage{array}

\usepackage{indentfirst}
\usepackage{xcolor,graphicx, tikz, wrapfig}
\usepackage{longtable}

\usepackage{minted}
\usemintedstyle{vs}

\usepackage[left=2cm,right=2cm,top=2cm,bottom=2cm,bindingoffset=0cm]{geometry}

\usepackage[unicode]{hyperref}
\definecolor{linkcolor}{HTML}{0000E6}
\definecolor{urlcolor}{HTML}{0000E6}
\definecolor{citecolor}{HTML}{0000E6}
\hypersetup{pdfpagemode=None,linktoc=page,citecolor=citecolor,linkcolor=linkcolor,urlcolor=urlcolor,colorlinks=true}

\theoremstyle{definition}
\newtheorem{subtask}{Пункт}

\DeclareMathOperator*{\argmax}{arg\,max}
\DeclareMathOperator*{\argmin}{arg\,min}

\setlength{\parindent}{1cm}

\author{Клычков Максим Дмитриевич}

\begin{document}

\centerline{\textbf{\huge Алгоритмы и структуры данных-1}}
\centerline{\textbf{SET 1. Задача A3.}}
\begin{flushright}
	\emph{Осень 2024. Клычков М. Д.}
\end{flushright}

\begin{figure}[htp]
	\centering
	\begin{minted}[linenos]{cpp}
void NestedLoops(size_t n) {
  int x = 100;
  int y = 0;
  for (size_t outer = 1; outer <= n; outer *= 2) {
    x = x + outer;
    for (size_t inner = 2; inner < n; ++inner) {
      if (x > y / inner) {
        y = y + outer / inner;
      } else {
        --y;
      }
    }
  }
}
	\end{minted}
\end{figure}

\begin{subtask}
	Начнем исследование циклического алгоритма с внешнего цикла.

	Легко определить количество итераций цикла, оно зависит только от $n$. Итак, всего будет произведено $(\left\lfloor \log_2(n) \right\rfloor + 1)$ итераций и $(\left\lfloor \log_2(n) \right\rfloor + 2)$ сравнений внутри \texttt{for}.

	В теле внешнего цикла одно присваивание с арифметической операцией (итого $2$ операции) и еще один (вложенный) цикл.

	Теперь разберем внутренний цикл: всего будет произведено $(n - 2)$ итерации и $(n - 1)$ сравнение. В теле цикла лишь один условный оператор, который перед самим сравнением делает операцию деления.

	Заметим, что количество операций в каждой ветви \emph{отличается}:
	\begin{itemize}
		\item в ветви выполнения условия всего происходит $3$ атомарные операции (присваивание и $2$ арифметические операции)
		\item в ветви невыполнения условия всего происходит 2 операции (присваивание и уменьшение на единицу)
	\end{itemize}

	Достаточно трудно определить, количество вхождений в каждую ветвь, так как при недостаточно больших $n$, которые возможно перебрать на бумаге условие конструкции \texttt{if} выполняется всегда. Предпримем попытку усреднить значения на примере конкретных $n$.

	Для этого с посчитаем для каждого $n$ количество вхождений в каждую из ветвей.

	\begin{longtable}{|>{$}l<{$}|>{$}l<{$}|>{$}l<{$}|}
		\hline
		n         & \texttt{true} & \texttt{false}       \\ \hline
		\endfirsthead
		%
		\multicolumn{3}{c}%
		{{\bfseries Table continued from previous page}} \\
		\endhead
		%
		1         & 0             & 0                    \\ \hline
		2         & 0             & 0                    \\ \hline
		4         & 6             & 0                    \\ \hline
		8         & 24            & 0                    \\ \hline
		16        & 70            & 0                    \\ \hline
		32        & 180           & 0                    \\ \hline
		64        & 434           & 0                    \\ \hline
		128       & 1008          & 0                    \\ \hline
		256       & 2286          & 0                    \\ \hline
		512       & 5100          & 0                    \\ \hline
		1024      & 11241         & 1                    \\ \hline
		2048      & 24550         & 2                    \\ \hline
		4096      & 53219         & 3                    \\ \hline
		8192      & 114655        & 5                    \\ \hline
		16384     & 245723        & 7                    \\ \hline
		32768     & 524247        & 9                    \\ \hline
		65536     & 1114066       & 12                   \\ \hline
		131072    & 2359245       & 15                   \\ \hline
		262144    & 4980680       & 18                   \\ \hline
		524288    & 10485699      & 21                   \\ \hline
		1048576   & 22020029      & 25                   \\ \hline
		2097152   & 46137271      & 29                   \\ \hline
		4194304   & 96468913      & 33                   \\ \hline
		8388608   & 201326506     & 38                   \\ \hline
		16777216  & 419430307     & 43                   \\ \hline
		33554432  & 872415132     & 48                   \\ \hline
		67108864  & 1811939220    & 54                   \\ \hline
		134217728 & 3758096268    & 60                   \\ \hline
		268435456 & 7784628100    & 66                   \\ \hline
	\end{longtable}

	Сравним число \texttt{true} (выполнения условия конструкции \texttt{if}) с \texttt{false} (невыполнения), заметим, что отношение этих чисел имеет порядок хотя бы $10^{-5}$, так что это отличие пренебрежительно мало. Более того, несрабатывание условия \texttt{if} происходит уже на больших числах, так что отличие в одной операции между двумя ветвями компенсируется сложностью арифметических операций, которая растет с ростом порядка числа. Учитывая все вышеперечисленные факты, будем считать, что в теле внутреннего цикла выполняются следующие операции:

	$$
		\underbrace{2}_{\texttt{if header}} + \underbrace{3}_{\texttt{if body}} = 5
	$$

	Подведем итог:
	$
		T(n) = \underbrace{(\left\lfloor \log_2(n) \right\rfloor + 2)}_{\texttt{loop compares}} + (\left\lfloor \log_2(n) \right\rfloor + 1) \cdot (2 + \underbrace{(n - 1)}_{\texttt{loop compares}} + (n - 2) \cdot 5)
	$
	\bigskip

	\begin{equation} \label{eq:1}
		T(n) = 1 + (\left\lfloor \log_2(n) \right\rfloor + 1) (6n - 9)
	\end{equation}
\end{subtask}

\begin{subtask}
	Хотим доказать, что $T(n) = \Theta(n \log n)$.

	Согласно определению, нужно найти такие $c_{1}$, $c_{2}$ и $N_{0}$, что:
	$$
		\forall n > N_0 \colon c_{1} n \log_2 n \overset{(2)}{\le}
		1 + (\left\lfloor \log_2(n) \right\rfloor + 1) (6n - 9)
		\overset{(3)}{\le} c_{2} n \log_2 n
	$$

	$(2)$:
	\begin{align*}
		T(n) & = 1 + (\left\lfloor \log_2(n) \right\rfloor + 1) (6n - 9) =                                     \\
		     & = 6n \left\lfloor \log_2(n) \right\rfloor - 9 \left\lfloor \log_2(n) \right\rfloor + 6n - 8 \ge \\
		     & \ge 6n (\log_2(n) - 1) - 9 (\log_2(n) - 1) + 6n - 8 =                                           \\
		     & = 6n \log_2(n) - 9 \log_2(n) + 1 \ge c_{1} n \log_2 n
	\end{align*}

	$$
		6 - \frac{9}{n} + \frac{1}{n \log_2 n} \ge c_{1} \Rightarrow c_{1} = 6 \wedge N_{0} = 32
	$$
	\pagebreak

	$(3)$:
	\begin{align*}
		T(n) & = 1 + (\left\lfloor \log_2(n) \right\rfloor + 1) (6n - 9) =                                     \\
		     & = 6n \left\lfloor \log_2(n) \right\rfloor - 9 \left\lfloor \log_2(n) \right\rfloor + 6n - 8 \le \\
		     & \le 6n \log_2(n) + 6n \le c_{2} n \log_2(n)
	\end{align*}

	$$
		c_2 \ge 6 + \frac{6}{\log_2(n)} \Rightarrow c_{2} = 8 \wedge N_{0} = 32
	$$

	Итого:
	$$
		\forall n > 32 \colon 6 n \log_2(n) \le T(n) \le 8 n \log_2(n)
	$$
\end{subtask}

\end{document}