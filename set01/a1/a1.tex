\documentclass[11pt,a4paper]{scrarticle}
\usepackage[utf8]{inputenc}
\usepackage{cmap}
\usepackage[T2A]{fontenc}
\usepackage[russian]{babel}
\usepackage{amsmath,amssymb,amsthm}

\usepackage{indentfirst}
\usepackage{xcolor,graphicx, tikz, wrapfig}

\usepackage{minted}
\usemintedstyle{vs}

\usepackage[left=2cm,right=2cm,top=2cm,bottom=2cm,bindingoffset=0cm]{geometry}

\usepackage[unicode]{hyperref}
\definecolor{linkcolor}{HTML}{0000E6}
\definecolor{urlcolor}{HTML}{0000E6}
\definecolor{citecolor}{HTML}{0000E6}
\hypersetup{pdfpagemode=None,linktoc=page,citecolor=citecolor,linkcolor=linkcolor,urlcolor=urlcolor,colorlinks=true}

\theoremstyle{definition}
\newtheorem{subtask}{Пункт}

\DeclareMathOperator*{\argmax}{arg\,max}
\DeclareMathOperator*{\argmin}{arg\,min}

\setlength{\parindent}{1cm}

\author{Клычков Максим Дмитриевич}

\begin{document}

\centerline{\textbf{\huge Алгоритмы и структуры данных-1}}
\centerline{\textbf{SET 1. Задача A1.}}
\begin{flushright}
\emph{Осень 2024. Клычков М. Д.}
\end{flushright}

\begin{figure}[htp]
	\centering
	\inputminted[linenos]{cpp}{selection_sort.cpp}
\end{figure}

\begin{subtask}
	В качестве инварианта внутреннего цикла по $j$ можно выбрать условие:
	$$P_{1} := \{minId = \argmin_{i \le k < j} A_k\}$$
	
	Приведем краткое обоснование на словах:
	
	Очевидно, что перед заходом в цикл инвариант выполняется: $\argmin (\{A_i\}) = i = minId$. В теле цикла находится лишь одна конструкция \texttt{if}, условие которой выполняется только при $A_{j} < A_{minId}$. Если утверждение верно, то $min(\{A_i, \dots, A_j\}) = A_{j}$, иначе $min(\{A_i, \dots, A_j\}) = A_{minId}$. После инкремента $j$ снова получается, что $\displaystyle minId = \argmin_{i \le k < j} A_k$, то есть инвариант выполняется.
	
	Такой вот индукцией мы кратко доказали, что $P_1$ подходит в качестве инварианта внутреннего цикла. На самом деле некоторое неявное подобие доказательства с \texttt{INIT}, \texttt{MNT} и \texttt{TRM}, которое будет описано далее.
\end{subtask}

\begin{subtask}
	В качестве инварианта внешнего цикла по $i$ можно выбрать условие: \\
	$P_2 :=$ подмассив $\{A_k \mid k \in [0; i)\}$ состоит из $i$ наименьших элементов начального массива, упорядоченных в порядке возрастания.
	
	Краткое обоснование:
	
	В теле внешнего цикла инициализируется переменная $minId$, которая после исполнения внутреннего цикла будет содержать индекс наименьшего элемента на хвосте $A[i \dots n - 1]$. Теперь мы ставим на позицию $i$ найденный минимум, который стоит на позиции $minId$. Так, для каждого $i$ мы находим $(i + 1)$-ый минимум массива и ставим его на то место, которое он будет занимать в отсортированном массиве. После инкремента выполняется инвариант: подмассив $\{A_k \mid k \in [0; i)\}$ состоит из $i$ наименьших элементов начального массива, упорядоченных в порядке возрастания.
	
	В конце, когда $i = n$, получаем, что весь массив отсортирован по возрастанию.  
\end{subtask}

\begin{subtask}
	Разберем сначала внутренний цикл, а затем внешний.
	
	\textbf{Внутренний цикл:}
	
	\texttt{Инвариант:} $\displaystyle P_{1} := \{minId = \argmin_{i \le k < j} A_k\}$.
	
	\texttt{INIT:} Тривиальный случай: $\argmin (\{A_i\}) = i = minId$.
	
	\texttt{MNT:} В теле цикла находится лишь одна конструкция \texttt{if}, условие которой выполняется только при $A_{j} < A_{minId}$.
	
	Если утверждение верно, то $min(\{A_i, \dots, A_j\}) = A_{j}$, так как
	$$min (\{A_i, \dots, A_{j - 1}\}) = A_{minId} \land A_{j} < A_{minId}$$
	Теперь $minId = j$, после инкремента $j$ снова получается, что $\displaystyle minId = \argmin_{i \le k < j} A_k$, то есть инвариант выполняется.
	
	Если утверждение неверно, то $min(\{A_i, \dots, A_j\}) = A_{minId}$, так как
	$$min (\{A_i, \dots, A_{j - 1}\}) = A_{minId} \land A_{j} >= A_{minId}$$
	Остается, что $minId = j$, после инкремента $j$ снова получается, что $\displaystyle minId = \argmin_{i \le k < j} A_k$, то есть инвариант выполняется.
	
	\texttt{TRM:} Причиной окончания цикла является нарушение условия $j < n$. Так как на каждом шаге переменная $j$ увеличивалась на 1, то в конце работы цикла $j = n$. Тогда
	$$minId = \argmin_{i \le k < n} A_k$$
	что является индексом $(i + 1)$-го минимума всего массива.
	
	\textbf{Внешний цикл:}
	
	\texttt{Инвариант:} $P_2 :=$ подмассив $\{A_k \mid k \in [0; i)\}$ состоит из $i$ наименьших элементов начального массива, упорядоченных в порядке возрастания.
	
	\texttt{INIT:} Тривиальный случай: массив $\{A_k \mid k \in [0; i)\}$ размера $0$ упорядочен.
	
	\texttt{MNT:} В теле цикла найдется наименьший элемент на хвосте $A[i \dots n - 1]$ (согласно доказательству инварианта внутреннего цикла). Теперь мы ставим на позицию $i$ найденный минимум, который стоит на позиции $minId$. Этот минимум будет являться $(i + 1)$ (до инкремента), так как на предыдущих итерациях уже были найдены $i$ меньших значений. Таким образом, после инкремента верно, что начало массива $\{A_k \mid k \in [0; i)\}$ состоит из $i$ наименьших элементов начального массива, упорядоченных в порядке возрастания.
	
	\texttt{TRM:} Причиной окончания цикла является нарушение условия $i < n$. Так как на каждом шаге переменная $i$ увеличивалась на 1, то в конце работы цикла $i = n$. Тогда полученный после последней итерации исходный массив состоит из $n$ наименьших элементов (то есть всех, так как общее их количество $= n$), упорядоченных по возрастанию, то есть весь массив отсортирован.
	
\end{subtask}

\end{document} 